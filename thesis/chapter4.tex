\chapter{Experimental results} \label{chap:results}

\section*{}
This chapter provides the results of the experiments conducted. The setup and
configurations are also described. There is a public page with the experimental
results of Schwa on Github \footnote{\url{https://github.com/andrefreitas/schwa/wiki/Experiments}}.

\section{Features weight estimation}
In this section we present the results of learning the features weights, that is
the importance of each of them, when computing the defect probability.

\subsection{Experimental setup}
To run this experiment, Schwa was invoked in the learning mode with 5, 50 and
100 commits with the script in listing 4.1. The version of Schwa used was
0.1.dev24(tagged on Git).

Running genetic algorithms takes a substantial time of computation, so we have
used Crowdsourcing to run the experiment in a variety of projects. We collected
data from academic, enterprise and Open Source projects to have results from
different contexts.

\begin{lstlisting}[language=bash, caption=Shell script used to learn features weights]
if [ -z "$1" ]
then
  echo "usage: $0 <git repository path>"
else
  touch report_$USER.txt
  echo "This will take a while..."
  echo "Learning with 5 commits"
  schwa $1 --commits 5 -l >> report_$USER.txt
  echo "Learning with 50 commits"
  schwa $1 --commits 50 -l >> report_$USER.txt
  echo "Learning with 100 commits"
  schwa $1 --commits 100 -l >> report_$USER.txt
  echo "Thank you! You are the best! Send report_$USER.txt to Andre :)"
fi
\end{lstlisting}

\subsection{Results}
Here we show the results for each repository. In the following tables, in each
row we have the maximum number of commits, the weights of revisions, fixes and
authors and the fitness value. The main objective of these tables is providing
the importance of each feature for a certain project.

\subsubsection{Schwa}
The experiment was run in
Schwa\footnote{\url{https://github.com/andrefreitas/schwa}} itself, that have
only one contributor. It is developed mostly with Python along with HTML, CSS
and Javascript.

\begin{table}[H]
    \centering
    \caption{Schwa results}
    \label{table:learning_schwa}
    \begin{tabular}{|l|l|l|l|l|}
        \hline
        Commits & Revisions & Fixes & Authors & Fitness \\ \hline
        5 & 0.2857 & 0.2857 & 0.4286 & 0 \\ \hline
        50 & 0.7143 & 0.1429 & 0.1429 & 1.0699 \\ \hline
        100 & 0.7143 & 0.1429 & 0.1429 & 1.1875 \\ \hline
    \end{tabular}
\end{table}

For 5 commits, the fitness function is 0, so it is not possible to estimate
correctly the weights. For 50 and 100 commits, revisions is the most important
feature.

\subsubsection{Libcrowbar}
Libcrowbar is the main repository of Crowbar and have multiple academic
contributors. The technologies used are Java, C++, HTML, CSS and Javascript.

\begin{table}[H]
    \centering
    \caption{Libcrowbar results}
    \label{table:learning_libcrowbar}
    \begin{tabular}{|l|l|l|l|l|}
        \hline
        Commits & Revisions & Fixes & Authors & Fitness \\ \hline
        5 & 0.1429 & 0.1429 & 0.7143 & 0.3486 \\ \hline
        50 & 0.7143 & 0.1429 & 0.1429 & 1.3639 \\ \hline
        100 & 0.7143 & 0.1429 & 0.1429 & 0.3387 \\ \hline
    \end{tabular}
\end{table}
For the last 5 commits, authors is the most important feature and for 50 and 100
commits, is revisions.

\subsubsection{Joda Time}
Joda Time\footnote{\url{https://github.com/JodaOrg/joda-time}} is an Open Source
time library for Java with multiple contributors.

\begin{table}[H]
    \centering
    \caption{Joda Time results}
    \label{table:learning_jodatime}
    \begin{tabular}{|l|l|l|l|l|}
        \hline
        Commits & Revisions & Fixes & Authors & Fitness \\ \hline
        5 & 0.1429 & 0.2857 & 0.5714 & 0 \\ \hline
        50 & 0.1429 & 0.7143 & 0.1429 & -2.1874 \\ \hline
        100 & 0.1429 & 0.7143 & 0.1429 & 0.3893 \\ \hline
    \end{tabular}
\end{table}
For this project only for 100 commits the fitness function is > 0, showing that
fixes is the most important feature.

\subsubsection{Meo Arena}
Meo Arena\footnote{\url{https://bitbucket.org/andrefreitas/feup-cmov-meoarena/}}
is a mobile application developed for Android in an academic context with two
contributors. It relies on Java and Python.

\begin{table}[H]
    \centering
    \caption{Meo Arena results}
    \label{table:learning_meoarena}
    \begin{tabular}{|l|l|l|l|l|}
        \hline
        Commits & Revisions & Fixes & Authors & Fitness \\ \hline
        5 & 0.2857 & 0.4286 & 0.2857 & 0 \\ \hline
        50 & 0.7143 & 0.1429 & 0.1429 & 1.6712 \\ \hline
        100 & 0.7143 & 0.1429 & 0.1429 & 0.7368 \\ \hline
    \end{tabular}
\end{table}

For 5 commits, fitness is equal to zero. For 50 and 100 commits the most
important feature is revisions.

\subsubsection{Mongo Java Driver}
Mongo Java driver\footnote{\url{https://github.com/mongodb/mongo-java-driver}}
is an Open Source projects and enables Java applications to communicate with a
MongoDB server. It is developed with Java and Groovy and has multiple
contributors.

\begin{table}[H]
    \centering
    \caption{Mongo Java driver results}
    \label{table:learning_mongojavadriver}
    \begin{tabular}{|l|l|l|l|l|}
        \hline
        Commits & Revisions & Fixes & Authors & Fitness \\ \hline
        5 & 0.7143 & 0.1429 & 0.1429 & 0.3486 \\ \hline
        50 & 0.7143 & 0.1429 & 0.1429 & 0.1685 \\ \hline
        100 & 0.1429 & 0.7143  & 0.1429 & 1.4666 \\ \hline
    \end{tabular}
\end{table}

For 5 and 50 commits the most important feature is revisions but for 100, is
fixes.

\subsubsection{Scraim}
Scraim\footnote{\url{https://scraim.com/}} is a web-based project management
tool developed by the company Strongstep. It is build on top of Redmine and
uses Ruby on Rails.

\begin{table}[H]
    \centering
    \caption{Scraim results}
    \label{table:learning_scraim}
    \begin{tabular}{|l|l|l|l|l|}
        \hline
        Commits & Revisions & Fixes & Authors & Fitness \\ \hline
        5         & 0.4286     & 0.1429 & 0.4286   & 0.3486 \\ \hline
        50        & 0.1429 & 0.1429 & 0.7143   & 0.3146 \\ \hline
        100       & 0.1429     & 0.7143  & 0.1429    & 0.9399 \\ \hline
    \end{tabular}
\end{table}
For 5 commits, revisions and authors are both important. For 50, authors is the
most important and for 100 is fixes.

\subsubsection{Trainsim}
Trainsim is an academic project developed with Java and Swing with one
contributor.

\begin{table}[H]
    \centering
    \caption{Trainsim results}
    \label{table:learning_trainsim}
    \begin{tabular}{|l|l|l|l|l|}
        \hline
        Commits & Revisions & Fixes & Authors & Fitness \\ \hline
        5         & 0.1429     & 0.7143 & 0.1429   & 0 \\ \hline
        50        & 0.7143 & 0.1429 & 0.1429   & 1.5945 \\ \hline
        100       & 0.5714     & 0.2857  & 0.1429    & 2.0425 \\ \hline
    \end{tabular}
\end{table}

For 5 commits fitness is zero, for 50 and 100 the most important is revisions.
Fixes importance increased from 50 to 100 commits.

\subsubsection{ShiftForward}
ShifForward\footnote{\url{http://shiftforward.eu}} is a company that develops
advertising technology with Scala. They have contributed with results of three
projects.

\begin{table}[H]
    \centering
    \caption{Adstax results}
    \label{table:learning_adstax}
    \begin{tabular}{|l|l|l|l|l|}
        \hline
        Commits & Revisions & Fixes & Authors & Fitness \\ \hline
        5         & 0.4286     & 0.1429 & 0.4286   & 0 \\ \hline
        50        & 0.4286 & 0.4286 & 0.1429   & 1.9044 \\ \hline
        100       & 0.2857     & 0.4286  & 0.2857   & 3.3134 \\ \hline
    \end{tabular}
\end{table}

In the Adstax project, for 5 commits the fitness is zero. For 50 commits,
revisions and fixes are both important and for 100, fixes is the most important.

\begin{table}[H]
    \centering
    \caption{Boxer results}
    \label{table:learning_boxer}
    \begin{tabular}{|l|l|l|l|l|}
        \hline
        Commits & Revisions & Fixes & Authors & Fitness \\ \hline
        5         & 0.2857     & 0.1429 & 0.5714   & 0.4408 \\ \hline
        50        & 0.1429 & 0.1429 & 0.7143   & 0.4457  \\ \hline
        100       & 0.1429     & 0.7143  & 0.1429   & -1.4159 \\ \hline
    \end{tabular}
\end{table}

In the Boxer project, for 5 and 50 commits, authors is the most important
feature. For 100 commits, the fitness is negative.

\begin{table}[H]
    \centering
    \caption{Apso results}
    \label{table:learning_apso}
    \begin{tabular}{|l|l|l|l|l|}
        \hline
        Commits & Revisions & Fixes & Authors & Fitness \\ \hline
        5         & 0.7143     & 0.1429 & 0.1429   & 0.9439 \\ \hline
        50        & 0.1429 & 0.7143 & 0.1429   & 1.9547 \\ \hline
        100       & 0.1429     & 0.4286  & 0.4286   & 1.3927 \\ \hline
    \end{tabular}
\end{table}

In the project Apso, revisions is the most important feature for 5 commits. For
50 commits is fixes and for 100 commits is fixes and authors.

\subsubsection{Hivedb}
Hivedb\footnote{\url{http://hivedb.org}} is an Open Source framework for
horizontally partitioning MySQL systems. It is developed mostly in Java with
multiple contributors.

\begin{table}[H]
    \centering
    \caption{Hivedb results}
    \label{table:learning_hivedb}
    \begin{tabular}{|l|l|l|l|l|}
        \hline
        Commits & Revisions & Fixes & Authors & Fitness \\ \hline
        5         & 0.7143     & 0.1429 & 0.1429   & 0  \\ \hline
        50        & 0.1429 & 0.7143 & 0.1429   & 0.1739 \\ \hline
        100       & 0.7143     & 0.1429  & 0.1429   & 0.9095 \\ \hline
    \end{tabular}
\end{table}

For 5 commits the fitness function is zero. For 50 commits the most important
feature is fixes and for 100, is revisions.

\subsubsection{Teamengine}
Teamengine\footnote{\url{https://github.com/opengeospatial/teamengine}} is an
Open Source engine to test web services and other resources in Java.

\begin{table}[H]
    \centering
    \caption{Teamengine results}
    \label{table:learning_teamengine}
    \begin{tabular}{|l|l|l|l|l|}
        \hline
        Commits & Revisions & Fixes & Authors & Fitness \\ \hline
        5         & 0.7143     & 0.1429 & 0.1429   & 0  \\ \hline
        50        & 0.1429 & 0.7143 & 0.1429   & 1.1356 \\ \hline
        100       & 0.7143     & 0.1429  & 0.1429   & 3.1080 \\ \hline
    \end{tabular}
\end{table}

For 5 commits the fitness function is 0. Fixes is the most important feature for
50 commits and for 100 is revisions.

\subsubsection{Automatalib}
Automatalib\footnote{\url{https://github.com/misberner/automatalib}} is an Open
Source Java Library for representing automata, graphs and transition systems.

\begin{table}[H]
    \centering
    \caption{Automatalib results}
    \label{table:learning_automatalib}
    \begin{tabular}{|l|l|l|l|l|}
        \hline
        Commits & Revisions & Fixes & Authors & Fitness \\ \hline
        5         & 0.1429     & 0.1429 & 0.7143   & 0.3486 \\ \hline
        50        & 0.1429 & 0.7143 & 0.1429   & -0.1952 \\ \hline
        100       & 0.4286     & 0.4286  & 0.1429   & 0.5261 \\ \hline
    \end{tabular}
\end{table}

For 5 commits, authors is the most important feature. For 50 commits the fitness
is negative. Revisions and fixes have the same weight for 100 commits.

\subsubsection{CDI TCK}
CDI TCK\footnote{\url{https://github.com/cdi-spec/cdi-tck}} is an Open Source
Context and Dependency Injection for Java EE developed with Java.

\begin{table}[H]
    \centering
    \caption{CDI TCK results}
    \label{table:learning_cditck}
    \begin{tabular}{|l|l|l|l|l|}
        \hline
        Commits & Revisions & Fixes & Authors & Fitness \\ \hline
        5         & 0.1429     & 0.5714 & 0.2857   & 0 \\ \hline
        50        & 0.2857 & 0.2857 & 0.4286   & 0 \\ \hline
        100       & 0.1429     & 0.7143  & 0.1429   & -0.3247 \\ \hline
    \end{tabular}
\end{table}

For 5 and 50 commits fitness is zero and for 100 is negative.

\section{Diagnostic cost}
In this section, the results of computing the diagnostic cost for each
configuration of Schwa in Crowbar are presented. In this experiment, the
computational cost is also substantial and crowdsourcing could not be used
since we need to inspect the behaviour of Crowbar with Schwa. Considering that
these experiments were run in a laptop (can last hours) with the constraints of
finding Java projects that use Git, have tests and support Maven, this phase
took weeks to find conclusive results.

Joda Time and CDI TCK projects were selected to run these experiment since they
have a Git repository and are Java projects with tests, so they both can be used
with Crowbar.

\subsection{Experimental setup}
For each project we had setup the experimental environment with the following
steps:

\begin{itemize}
\item Compute the weights for revisions, fixes and authors with Schwa learning
mode;
\item Create a .schwa.yml in the root of the repository with the weights and
maximum commits;
\item Insert bugs (e.g. wrong comparison) in methods and commit the changes;
\item Evaluate the diagnostic cost for using Schwa with priors, goodnesses or
both.
\end{itemize}


\subsection{Results}
The results are presented with the history of commits and configurations of
Schwa.

\subsubsection{Joda Time}
The sequence of commits applied in Joda Time is available on table
\ref{table:commits_jodatime} along with the commits that inserted bugs.

\begin{table}[H]
    \centering
    \caption{Commits applied to Joda Time}
    \label{table:commits_jodatime}
    \begin{tabular}{|c|l|l|}
        \hline
        Order & Commit & Description \\ \hline
        1 & 8207a55	& Added a defect in DateTime.java in withZoneRetainfields() \\ \hline
        2 & 74149c0 & Added a defect in Duration.java in minus() \\ \hline
        3 & 22a5f71 & Fixed withZoneRetainfields() bug \\ \hline
        4 & 0945c34 & Fixed minus() bug and added another bug \\ \hline
        5 & 92adf94 & Fixed previous bug and added one in getMaximumValue() \\ \hline

    \end{tabular}
\end{table}

\begin{lstlisting}[language=java, caption=Commit 8207a55 patch]
  public DateTime withZoneRetainFields(DateTimeZone newZone) {
         newZone = DateTimeUtils.getZone(newZone);
         DateTimeZone originalZone = DateTimeUtils.getZone(getZone());
-        if (newZone == originalZone) {
+        if (newZone != originalZone) {
             return this;
         }
\end{lstlisting}

\begin{lstlisting}[language=java, caption=Commit 74149c0 patch]
 public Duration minus(long amount) {
-        return withDurationAdded(amount, -1);
+        return withDurationAdded(amount, -2);
     }

   public Duration minus(ReadableDuration amount) {
-        if (amount == null) {
+        if (amount != null) {
             return this;
         }
         return withDurationAdded(amount.getMillis(), -1);
     }
\end{lstlisting}

\begin{lstlisting}[language=java, caption=Commit 92adf94 patch]
final class BasicDayOfMonthDateTimeField extends PreciseDurationDateTimeField {
                 int month = values[i];
                 for (int j = 0; j < size; j++) {
                     if (partial.getFieldType(j) == DateTimeFieldType.year()) {
-                        int year = values[j];
+                        int year = values[i];
                         return iChronology.getDaysInYearMonth(year, month);
                     }
                 }
\end{lstlisting}

The configurations used are in table \ref{table:configs_jodatime}. Schwa was
used with the default configuration values in configuration \#1. In the
configuration \#2 the weights learned from the genetic algorithms are used along
with the time range(tr=0.6), to increase the defect probabilities of the last
changed components.

\begin{table}[H]
    \centering
    \caption{Schwa configurations for Joda Time}
    \label{table:configs_jodatime}
    \begin{tabular}{|c|c|c|c|c|c|}
        \hline
        Configuration & Commits & Revisions & Fixes & Authors & Time Range \\ \hline
        \#1 & 20 & 0.25 & 0.5 & 0.25 & 0\\ \hline
        \#2 & 5 & 0.15 & 0.7 & 0.15 & 0.6\\ \hline
    \end{tabular}
\end{table}


The results for the diagnostic cost are presented in table
\ref{table:cd_jodatime} and they are grouped in scenarios. A scenario is a unique
configuration and revision (commit).


\begin{table}[H]
    \centering
    \caption{Diagnostic cost for Joda Time}
    \label{table:cd_jodatime}
    \begin{tabular}{|c|c|c|c|}
        \hline
        Schwa & Commit & Configuration & Diagnostic Cost \\ \hline
        \multicolumn{4}{|c|}{First scenario} \\ \hline
        None & 74149c0 & \#1 & 0 \\ \hline
        Both & 74149c0 & \#1 & 1 \\ \hline
        Priors & 74149c0 & \#1 & 1 \\ \hline
        Goodnesses & 74149c0 & \#1 & 1 \\ \hline
        \multicolumn{4}{|c|}{Second scenario} \\ \hline
        None & 22a5f71 & \#1 & 0 \\ \hline
        Both & 22a5f71 & \#1 & 0 \\ \hline
        Priors & 22a5f71 & \#1 & 1 \\ \hline
        Goodnesses & 22a5f71 & \#1 & 0 \\ \hline
        \multicolumn{4}{|c|}{Third scenario} \\ \hline
        None & 92adf94 & \#2 & 20 \\ \hline
        Both & 92adf94 & \#2 & 21 \\ \hline
        Priors & 92adf94 & \#2 & 20 \\ \hline
        Goodnesses & 92adf94 & \#2 & 21 \\ \hline

    \end{tabular}
\end{table}

As presented in table \ref{table:cd_jodatime}, using configuration \#1 in the
first scenario, when Schwa is used the diagnostic cost increases to 1. In the
second scenario, by fixing one of the bugs, the diagnostic cost only increases
when Schwa is used for priors. In the third scenario, by fixing the previous
bugs, add another one and using the time range parameter (new configuration),
the diagnostic cost increases for Both and Goodnesses, but is the same for Priors.

For the third scenario, we also evaluated what would be the maximum diagnostic
cost if Schwa would give optimal results. The defect probability of the faulty
component was set to 0.9, and for the healthy components was set to 0.1 (note
that this probabilities are only for components involved in the last commits
analyzed by Schwa). By evaluating the rank again, the diagnostic cost was 20
for priors and 21 for goodnesses and both.

\subsubsection{CDI TCK}
The commit applied in CDI TCK is on Table \ref{table:commits_cditck}.
\begin{table}[H]
    \centering
    \caption{Commits applied to CDI TCK}
    \label{table:commits_cditck}
    \begin{tabular}{|l|l|}
        \hline
        Commit & Description \\ \hline
        1bc2e2d	& Added a defect in ActionSequence.java toString() method \\ \hline

    \end{tabular}
\end{table}

\begin{lstlisting}[language=java, caption=Commit 1bc2e2d patch]
   @Override
     public String toString() {
-        return String.format("ActionSequence [name=%s, data=%s]", name, getData());
+        return String.format("ActionSequenc [name=%s, data=%s]", name, getData());
     }
\end{lstlisting}

\begin{table}[H]
    \centering
    \caption{Schwa configurations for CDI TCK}
    \label{table:configs_cdi_tck}
    \begin{tabular}{|c|c|c|c|c|c|}
        \hline
        Configuration & Commits & Revisions & Fixes & Authors & Time Range \\ \hline
        \#1 & 20 & 0.15 & 0.7 & 0.15 & 0\\ \hline
        \#2 & 20 & 0.7 & 0.15 & 0.15 & 0\\ \hline
        \#3 & 20 & 1 & 0 & 0 & 0\\ \hline
        \#4 & 5 & 1 & 0 & 0 & 0.6\\ \hline
        \#5 & 5 & 0.15 & 0.7 & 0.15 & 0.6\\ \hline
    \end{tabular}
\end{table}

The results for the diagnostic cost are the following:
\begin{table}[H]
    \centering
    \caption{Diagnostic cost for CDI TCK}
    \label{table:cd_cditck}
    \begin{tabular}{|c|c|c|c|}
        \hline
        Schwa & Commit & Configuration & Diagnostic Cost \\ \hline
        \multicolumn{4}{|c|}{First scenario} \\ \hline
        None & 1bc2e2d & \#1 & 0 \\ \hline
        Both & 1bc2e2d & \#1 & 6 \\ \hline
        Priors & 1bc2e2d & \#1 & 8 \\ \hline
        Goodnesses & 1bc2e2d & \#1 & 0 \\ \hline

        \multicolumn{4}{|c|}{Second scenario} \\ \hline
        Both & 1bc2e2d & \#2 & 6 \\ \hline
        Priors & 1bc2e2d & \#2 & 7 \\ \hline
        Goodnesses & 1bc2e2d & \#2 & 0 \\ \hline

        \multicolumn{4}{|c|}{Third scenario} \\ \hline
        Both & 1bc2e2d & \#3 & 0 \\ \hline
        Priors & 1bc2e2d & \#3 & 7 \\ \hline
        Goodnesses & 1bc2e2d & \#3 & 0 \\ \hline

        \multicolumn{4}{|c|}{Fourth scenario} \\ \hline
        Both & 1bc2e2d & \#4 & 0 \\ \hline
        Priors & 1bc2e2d & \#4 & 9 \\ \hline
        Goodnesses & 1bc2e2d & \#4 & 0 \\ \hline

        \multicolumn{4}{|c|}{Fifth scenario} \\ \hline
        Both & 1bc2e2d & \#5 & 9 \\ \hline
        Priors & 1bc2e2d & \#5 & 1 \\ \hline
        Goodnesses & 1bc2e2d & \#5 & 1 \\ \hline
    \end{tabular}
\end{table}

The experiment in CDI TCK was conducted by trying a variety of configurations,
to evaluate their impact. For the applied patch, the diagnostic cost is zero
without the usage of Schwa. In the first scenario by giving more importance to
fixes, the diagnostic cost is worse for all options, except for goodnesses. In
the second scenario, by giving more importance to revisions, the results are
practically the same: worse for all options except goodnesses.

In the third and fourth scenarios, by giving only importance to revisions the
diagnostic cost is zero for goodnesses and both. The time range parameter was
changed in the fourth scenario.

In the fifth scenario, by combining the usage of time range and giving 0.15 for
revisions and authors and 0.7 for fixes, the diagnostic cost increased in all
options.
