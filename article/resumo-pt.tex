%-----------------------------------------------
% Template para criação de resumos de projectos/dissertação
% jlopes AT fe.up.pt,   Fri Jul  3 11:08:59 2009
%-----------------------------------------------

\documentclass[9pt,a4paper]{extarticle}

%% English version: comment first, uncomment second
\usepackage[portuguese]{babel}  % Portuguese
%\usepackage[english]{babel}     % English
\usepackage{graphicx}           % images .png or .pdf w/ pdflatex OR .eps w/ latex
\usepackage{times}              % use Times type-1 fonts
\usepackage[utf8]{inputenc}     % 8 bits using UTF-8
\usepackage{url}                % URLs
\usepackage{multicol}           % twocolumn, etc
\usepackage{float}              % improve figures & tables floating
\usepackage[tableposition=top]{caption} % captions
%% English version: comment first (maybe)
\usepackage{indentfirst}        % portuguese standard for paragraphs
%\usepackage{parskip}

%% page layout
\usepackage[a4paper,margin=30mm,noheadfoot]{geometry}

%% space between columns
\columnsep 12mm

%% headers & footers
\pagestyle{empty}

%% figure & table caption
\captionsetup{figurename=Fig.,tablename=Tab.,labelsep=endash,font=bf,skip=.5\baselineskip}

%% heading
\makeatletter
\renewcommand*{\@seccntformat}[1]{%
  \csname the#1\endcsname.\quad
}
\makeatother

%% avoid widows and orphans
\clubpenalty=300
\widowpenalty=300

\begin{document}

\title{\vspace*{-8mm}\textbf{\textsc{Software Repository Mining Analytics to Estimate Software Component Reliability}}}
\author{\emph{André Freitas - freitas.andre@fe.up.pt}\\[2mm]
\small{Dissertação realizada sob a orientação do \emph{Prof.\ Rui Maranhão e Alexandre Perez}}}
\date{}
\maketitle
%no page number
\thispagestyle{empty}

\vspace*{-4mm}\noindent\rule{\textwidth}{0.4pt}\vspace*{4mm}

\begin{multicols}{2}

\section{Motivação}\label{sec:motiva}
O Software desempenha um papel fundamental na nossa sociedade e na nossa rotina
diária, pois dependemos de aplicações para comunicar, gerir informação, etc.
O desenvolvimento de Software é relativamente complexo e o custo de
corrigir bugs representa até 90\% do custo do projeto \cite{Servant1}.

Os programadores utilizam ferramentas de controlo de versões para gerir o
histórico de alterações no código. Os repositórios de Software têm portanto
informação valiosa que pode ser explorada com técnicas de \emph{Machine Learning}
e de \emph{Analytics} para suportar modelos de previsão de defeitos de Software.

O Crowbar é uma ferramenta de localização automática de falhas, que após a
execução de uma bateria de testes, estima os componentes faltosos. O algoritmo
Barinel usado nesta ferramenta \footnote{http://crowbar.io} usa estimativas
estáticas. Estas podem ser substituídas por estimativas dinâmicas provenientes
do resultado da análise de repositórios de, de maneira a melhorar a qualidade do
diagnóstico de falhas no Crowbar.


\section{Objectivos}\label{sec:goals}

Os principais objetivos são:

\begin{itemize}
\item Prever defeitos a partir de repositórios de Software e aprender quais
são as variáveis mais importantes a analisar e criar um modelo de previsão
baseado nas técnicas existentes;
\item Melhorar o diagnóstico de falhas no Crowbar com base nos resultados da
previsão de defeitos.
\end{itemize}

\section{Descrição do Trabalho}\label{sec:work}
Foi implementada uma ferramenta com o nome Schwa, que está disponível livremente
no Github e que pode ser instalada a partir do gestor de pacotes de Python pelo
comando
```
pip install schwa --pre
```


\section{Conclusões}\label{sec:conclui}

%%English version: comment first, uncomment second
\bibliographystyle{unsrt-pt}  % numeric, unsorted refs
%\bibliographystyle{unsrt}  % numeric, unsorted refs
\bibliography{refs}

\end{multicols}

\end{document}
